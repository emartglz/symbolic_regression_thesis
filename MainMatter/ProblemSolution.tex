\chapter{Propuesta de solución}\label{chapter:solution_proposal}

En este capitulo se plantea una forma de obtener un sistema de ecuaciones diferenciales lineales en los parámetros mediante el uso de la regresión simbólica utilizando un algoritmo genético. De esta forma se obtiene un modelo que aproxima un conjunto de datos.

La primera sección del capítulo detalla como se puede representar una solución mediante un árbol computacional. En la sección siguiente se explica la función de ajuste que se tiene en cuenta en la regresión simbólica planteada. En las secciones 3, 4 y 5 se detallan las operaciones necesarias para la aplicación de un algoritmo genético: la mutación, el cruzamiento y la selección respectivamente.

\section{Cómo representar una solución}

Teniendo los datos de la forma $(ti, yi)$, se puede extraer información útil para conocer la forma que debe tener el modelo que se intenta encontrar con el algoritmo. Por ejemplo, si cada elemento de los datos es de la forma:

$$(t_i, y_{1_i}, y_{2_i}, y_{3_i})$$,

entonces se tiene la certeza de que el sistema de ecuaciones diferenciales que se desea tiene tres ecuaciones, y que los sistemas de ecuaciones en el espacio de búsqueda del método deben tener la forma:

$$y_1' = f_1(t, y_1, y_2, y_3)$$
$$y_2' = f_2(t, y_1, y_2, y_3)$$
$$y_3' = f_3(t, y_1, y_2, y_3)$$

A partir de los datos se puede decir que las soluciones candidatas a solución son una lista de tres elementos, donde en la posición $i$ está representada la función $fi(t,y_1,y_2,y_3)$.

Por ejemplo, en el modelo poblacional SIR:

\begin{align*}
    S' & = - a*I*S     \\
    I' & = a*I*S - b*I \\
    R' & = b*I,
\end{align*}

la lista estaría formada por

$$[-a*I*S, a*I*S - b*I, b*I].$$

Una expresión aritmética (como cada una de las funciones) se puede representar mediante un árbol, donde los nodos interiores son operadores y las hojas son variables. Entonces se puede representar en cada posición de la lista el árbol computacional que describe la parte de la derecha de la ecuación diferencial correspondiente a esa posición.

Sin embargo con esta representación no se plantea de forma explícita la linealidad de las ecuaciones diferenciales con respecto a los parámetros. Por lo que no es útil ya que en el espacio de búsqueda del algoritmo no deben aparecer modelos que incumplan esta propiedad. Para que esta forma de representar las funciones en forma de árboles computacionales no permita la no linealidad con respecto a los parámetros se modifica la estructura del árbol de la siguiente manera.

Como la parte derecha de una ecuación diferencial lineal con respecto a los parámetros es una sumatoria de multiplicaciones de parámetros con funciones que no dependen de parámetros, entonces cada una de las partes derechas de las ecuaciones diferenciales se representan con un árbol en el que la raiz es un nodo con una operación especial de suma, que puede tener cualquier cantidad de sumandos (o de hijos).

Cada uno de estos hijos es un subárbol que representa la multiplicación de un parámetro con una función que no depende de parámetros. Estos subárboles poseen como raiz un nodo con una operación de multiplicación y dos nodos hijos. El primero de ellos sería un nodo hoja representando el parámetro. El segundo de ellos sería un nodo que representa el subárbol computacional correspondiente a la función utilizando la misma representación planteada por Koza pero con la peculiaridad de que sus hojas solo podrán almacenar variables, no parámetros.

Cada ecuación diferencial sería un hijo de un nodo que representa al sistema de ecuaciones diferenciales. Como ejemplo se puede utilizar el modelo poblacional SIR que es lineal con respecto a sus parámetros, por lo que su represetación con la estructura planteada anteriormente sería:

\begin{center}
    \begin{tikzpicture}[
            roundnode/.style={circle, draw=red!60, fill=red!5, very thick, minimum size=7mm},
            squarednode/.style={rectangle, draw=red!60, fill=red!5, very thick, minimum size=5mm}
        ]
        % Nodes
        \node[roundnode]        (system)                            {$SYSTEM$};

        \node[roundnode]        (plus_S)     [below left=of system]        {$+$};
        \node[roundnode]        (star_S_1)    [below left=of plus_S]    {$*$};
        \node[squarednode]      (alpha_star_S_1)      [below left=of star_S_1]    {$a$};
        \node[roundnode]        (neg_star_S_1)    [below=of star_S_1]    {$-$};
        \node[roundnode]        (star_S_2)    [below=of neg_star_S_1]    {$*$};
        \node[squarednode]      (S_star_S)       [below left=of star_S_2]   {$S$};
        \node[squarednode]      (I_star_S)      [below=of star_S_2]   {$I$};

        \node[roundnode]        (plus_I)     [below=of system]        {$+$};
        \node[roundnode]        (star_I_1)    [below left=of plus_I]    {$*$};
        \node[squarednode]      (alpha_star_I_1)      [below left=1cm and 0.5cm of star_I_1]    {$a$};
        \node[roundnode]        (star_I_2)    [below=of star_I_1]    {$*$};
        \node[squarednode]      (S_star_I)       [below left=1cm and 0.5cm of star_I_2]   {$S$};
        \node[squarednode]      (I_star_I_1)      [below=of star_I_2]   {$I$};

        \node[roundnode]        (star_I_3)    [below right=of plus_I]    {$*$};
        \node[squarednode]      (beta_star_I_1)      [below=of star_I_3]    {$b$};
        \node[roundnode]        (neg_star_I_1)    [below right=1cm and 0.5cm of star_I_3]    {$-$};
        \node[squarednode]      (I_star_I_2)      [below=of neg_star_I_1]   {$I$};

        \node[roundnode]        (plus_R)     [below right=of system]        {$+$};
        \node[roundnode]        (star_R_1)    [below right=of plus_R]    {$*$};
        \node[squarednode]      (beta_star_R_1)      [below=of star_R_1]    {$b$};
        \node[squarednode]      (I_star_R)      [below right=of star_R_1]   {$I$};

        %Lines
        \draw [->] (system.south) -- (plus_S.north);
        \draw[->] (system.south) -- (plus_I.north);
        \draw[->] (system.south) -- (plus_R.north);

        \draw[->] (plus_S.south) -- (star_S_1.north);
        \draw[->] (star_S_1.south) -- (alpha_star_S_1.north);
        \draw[->] (star_S_1.south) -- (neg_star_S_1.north);
        \draw[->] (neg_star_S_1.south) -- (star_S_2.north);
        \draw[->] (star_S_2.south) -- (S_star_S.north);
        \draw[->] (star_S_2.south) -- (I_star_S.north);

        \draw[->] (plus_I.south) -- (star_I_1.north);
        \draw[->] (plus_I.south) -- (star_I_3.north);
        \draw[->] (star_I_1.south) -- (alpha_star_I_1.north);
        \draw[->] (star_I_1.south) -- (star_I_2.north);
        \draw[->] (star_I_2.south) -- (S_star_I.north);
        \draw[->] (star_I_2.south) -- (I_star_I_1.north);

        \draw[->] (star_I_3.south) -- (beta_star_I_1.north);
        \draw[->] (star_I_3.south) -- (neg_star_I_1.north);
        \draw[->] (neg_star_I_1.south) -- (I_star_I_2.north);

        \draw[->] (plus_R.south) -- (star_R_1.north);
        \draw[->] (star_R_1.south) -- (beta_star_R_1.north);
        \draw[->] (star_R_1.south) -- (I_star_R.north);
    \end{tikzpicture}
\end{center}

Con esta nueva representación se pueden expresar todos los sistemas de ecuaciones diferenciales lineales en los parámetros. Por esta razón es la estructura que se utiliza para la representación de los modelos generados por el algoritmo de regresión simbólica detallado en este capítulo.

En un método de regresión simbólica es necesario una función de ajuste que permita conocer que tan cercano es el modelo obtenido con respecto al modelo original, para esto se define el costo de un sistema que se explica en la sección siguiente.

\section{Determinar costo de una solución}

Se define el costo de una solución obtenida durante el método de la regresión simbólica como:

$$C = \frac{\sum_{i=1}^n\frac{\sum_{j=1}^{m}(y_{i_j} - f_i(x_j))^2}{m}}{n},$$

con:

$$f_i(x_j) = \sum_{k=1}^{p_i} a_{i_k} * g_{i_k}(x_j),$$

donde $f_1, f_2, \dots, f_n$ son las partes derechas de las ecuaciones que conforman la solución obtenida dados los puntos $(x_j, (y_{1_j}, y_{2_j}, \dots, y_{n_j}))$  con $1 < j < m$ siendo $m$ la cantidad de puntos, $p_i$ indica la cantidad de parámetros presentes en la ecuación $i$ del sistema.

Mientras más pequeño es el valor del costo, más se asemeja el sistema obtenido al sistema original ya que si es 0 significa que se ha encontrado un sistema que ajusta perfectamente los puntos.

Al valor de $C$ se le agrega un factor de peso, el cual es directamente propociornal a la cantidad de términos que posean las ecuaciones de la solución. Esto garantiza que si dos ecuaciones son capaces de generar los mismos puntos, la ecuación con menos términos posea menos peso.

Para que la diferencia $y_{i_j} - f_i(x_j)$ se la menor posible se ajustan los parámetros $a_{i_k}$ por cada una de las $f_i$. Para esto se crea, por cada ecuación del sistema, un sistema de ecuaciones de la forma $A_i * x_i = B_i$ donde:

\begin{align*}
    A_i & = \begin{pmatrix}
        g_{i_1}(x_1) & g_{i_2}(x_1) & \dots  & g_{i_k}(x_1) \\
        g_{i_1}(x_2) & g_{i_2}(x_2) &        & g_{i_k}(x_2) \\
        \vdots       & \vdots       & \ddots & \vdots       \\
        g_{i_1}(x_m) & g_{i_2}(x_m) &        & g_{i_k}(x_m)
    \end{pmatrix}
    \qquad
    x_i = \begin{bmatrix}
        a_{i_1} \\
        a_{i_2} \\
        \vdots  \\
        a_{i_k}
    \end{bmatrix}
    \qquad
    B_i = \begin{bmatrix}
        y_{i_1} \\
        y_{i_2} \\
        \vdots  \\
        y_{i_m}
    \end{bmatrix}.
\end{align*}

Este sistema se resuelve utilizando el método de ajuste mínimo cuadrático de datos, explicado en el capítulo anterior.

Por ejemplo, la segunda ecuación del modelo poblacional SIR analizado anteriormente es:

$$f_I (t,S,I,R) = a_{I_1} * g_{I_1}(t,S,I,R) + a_{I_2} * g_{I_2}(t,S,I,R),$$

donde:

$$g_{I_1}(t,S,I,R) = I*S$$

y

$$g_{I_2}(t,S,I,R) = -I.$$

Suponiendo que tenemos de entrada los puntos:

$$f_I(1, 4, 9, 6) = 5 $$
$$f_I(2, 6, 8, 2) = 4 $$
$$f_I(3, 3, 6, 8) = 10$$
$$f_I(4, 5, 9, 8) = 1 $$
$$f_I(5, 6, 3, 10) =7 $$
$$f_I(6, 8, 5, 7) = 6 $$

entonces se puede encontrar los valores de $a_1$ y $a_2$ que mejor ajusten esta ecuación con los puntos vistos anteriormente formando el sistema de ecuaciones:

\begin{align*}
    A_I & = \begin{pmatrix}
        36 & -9 \\
        48 & -8 \\
        18 & -6 \\
        45 & -9 \\
        18 & -3 \\
        40 & -5
    \end{pmatrix}
    \qquad
    x_I = \begin{bmatrix}
        a_{I_1} \\
        a_{I_2}
    \end{bmatrix}
    \qquad
    B_I = \begin{bmatrix}
        5  \\
        4  \\
        10 \\
        1  \\
        7  \\
        6
    \end{bmatrix}.
\end{align*}

Al resolver este sistema de ecuaciones sobredeterminados se obtienen los parámetros $a_{I_1}$ = -0.03570806 y $a_{I_2}$ = -0.84347764. De esta forma se encuentran los mejores parámetros para cada ecuación del sistema.

Como se plantea al inicio del capítulo, se propone como solución utilizar el método de regresión simbólica mediante el uso de un algoritmo genético para encontrar el sistema que mejor ajuste un conjunto de puntos. Entonces se necesitan definir las operaciones de mutación, cruzamiento y selección para poder utilizar esta metaheurística.

\section{Mutación}

La operación de mutación selecciona el subárbol representante de una ecuación aleatoria del sistema y luego se escoge una de las siguientes modificaciones:

\begin{itemize}
    \item Eliminar un término de la ecuación:

          \begin{center}
              \begin{adjustbox}{width=0.35\textwidth, keepaspectratio}
                  \begin{tikzpicture}[
                          roundnode_red/.style={circle, draw=red!60, fill=red!5, very thick, minimum size=7mm},
                          roundnode_yellow/.style={circle, draw=yellow!60, fill=yellow!5, very thick, minimum size=7mm},
                          roundnode/.style={circle, draw=red!60, fill=red!5, very thick, minimum size=7mm},
                          squarednode/.style={rectangle, draw=red!60, fill=red!5, very thick, minimum size=5mm},
                      ]
                      %Nodes
                      \node[roundnode_red]      (plus)                             {$+$};
                      \node[roundnode]           (star1)   [below left=of plus]    {$*$};
                      \node[squarednode]         (a_1)   [below left=of star1]    {$a_1$};
                      \node[squarednode]         (y_1)     [below=of star1]         {$y_1$};
                      \node[roundnode_yellow]           (star2)   [below right=of plus]   {$*$};
                      \node[squarednode]         (a_2)    [below left=of star2]    {$a_2$};
                      \node[roundnode]           (neg)     [below=of star2]         {$-$};
                      \node[roundnode]           (star3)   [below=of neg]         {$*$};
                      \node[squarednode]         (y_1_2)   [below=of star3]   {$y_1$};
                      \node[squarednode]         (y_2)     [below right=of star3]   {$y_2$};

                      %Lines
                      \draw[->] (plus.south) -- (star1.north);
                      \draw[->] (plus.south) -- (star2.north);
                      \draw[->] (star1.south) -- (a_1.north);
                      \draw[->] (star1.south) -- (y_1.north);
                      \draw[->] (star2.south) -- (a_2.north);
                      \draw[->] (star2.south) -- (neg.north);
                      \draw[->] (neg.south) -- (star3.north);
                      \draw[->] (star3.south) -- (y_1_2.north);
                      \draw[->] (star3.south) -- (y_2.north);
                  \end{tikzpicture}%
              \end{adjustbox}
              \qquad
              \begin{adjustbox}{width=0.20\textwidth, keepaspectratio}
                  \begin{tikzpicture}[
                          roundnode_red/.style={circle, draw=red!60, fill=red!5, very thick, minimum size=7mm},
                          roundnode_yellow/.style={circle, draw=yellow!60, fill=yellow!5, very thick, minimum size=7mm},
                          roundnode/.style={circle, draw=red!60, fill=red!5, very thick, minimum size=7mm},
                          squarednode/.style={rectangle, draw=red!60, fill=red!5, very thick, minimum size=5mm},
                      ]
                      %Nodes
                      \node[roundnode_red]      (plus)                             {$+$};
                      \node[roundnode]           (star1)   [below =of plus]    {$*$};
                      \node[squarednode]         (a_1)   [below left=of star1]    {$a_1$};
                      \node[squarednode]         (y_1)     [below right=of star1]         {$y_1$};

                      %Lines
                      \draw[->] (plus.south) -- (star1.north);
                      \draw[->] (star1.south) -- (a_1.north);
                      \draw[->] (star1.south) -- (y_1.north);
                  \end{tikzpicture}%
              \end{adjustbox}
          \end{center}

    \item Añadir un término a la ecuación:

          \begin{center}
              \begin{adjustbox}{width=0.25\textwidth, keepaspectratio}
                  \begin{tikzpicture}[
                          roundnode_red/.style={circle, draw=red!60, fill=red!5, very thick, minimum size=7mm},
                          roundnode_yellow/.style={circle, draw=yellow!60, fill=yellow!5, very thick, minimum size=7mm},
                          roundnode/.style={circle, draw=red!60, fill=red!5, very thick, minimum size=7mm},
                          squarednode/.style={rectangle, draw=red!60, fill=red!5, very thick, minimum size=5mm},
                      ]
                      %Nodes
                      \node[roundnode_red]      (plus)                             {$+$};
                      \node[roundnode]           (star1)   [below =of plus]    {$*$};
                      \node[squarednode]         (a_1)   [below left=of star1]    {$a_1$};
                      \node[squarednode]         (y_1)     [below right=of star1]         {$y_1$};

                      %Lines
                      \draw[->] (plus.south) -- (star1.north);
                      \draw[->] (star1.south) -- (a_1.north);
                      \draw[->] (star1.south) -- (y_1.north);
                  \end{tikzpicture}%
              \end{adjustbox}
              \qquad
              \begin{adjustbox}{width=0.35\textwidth, keepaspectratio}
                  \begin{tikzpicture}[
                          roundnode_red/.style={circle, draw=red!60, fill=red!5, very thick, minimum size=7mm},
                          roundnode_yellow/.style={circle, draw=yellow!60, fill=yellow!5, very thick, minimum size=7mm},
                          roundnode/.style={circle, draw=red!60, fill=red!5, very thick, minimum size=7mm},
                          squarednode/.style={rectangle, draw=red!60, fill=red!5, very thick, minimum size=5mm},
                      ]
                      %Nodes
                      \node[roundnode_red]      (plus)                             {$+$};
                      \node[roundnode]           (star1)   [below left=of plus]    {$*$};
                      \node[squarednode]         (a_1)   [below left=of star1]    {$a_1$};
                      \node[squarednode]         (y_1)     [below=of star1]         {$y_1$};
                      \node[roundnode_yellow]           (star2)   [below right=of plus]   {$*$};
                      \node[squarednode]         (a_2)    [below=of star2]    {$a_2$};
                      \node[squarednode]           (y_2)     [below right=of star2]         {$y_2$};

                      %Lines
                      \draw[->] (plus.south) -- (star1.north);
                      \draw[->] (plus.south) -- (star2.north);
                      \draw[->] (star1.south) -- (a_1.north);
                      \draw[->] (star1.south) -- (y_1.north);
                      \draw[->] (star2.south) -- (a_2.north);
                      \draw[->] (star2.south) -- (y_2.north);
                  \end{tikzpicture}%
              \end{adjustbox}
          \end{center}

    \item Mutar un término de la ecuación.
\end{itemize}

Si se escoge la mutación de un término de la ecuación en las opciones anteriores se pasa a seleccionar un nodo aleatorio dentro del subárbol que representa a $g_{i_k}$ y se aplica una de las siguientes modificaciones.

Si el nuevo nodo representa una operación:

\begin{itemize}
    \item Cambia la operación en el nodo por uno que posea la misma cantidad de parámetros del operador:

          \begin{center}
              \begin{adjustbox}{width=0.25\textwidth, keepaspectratio}
                  \begin{tikzpicture}[
                          roundnode_red/.style={circle, draw=red!60, fill=red!5, very thick, minimum size=7mm},
                          roundnode_yellow/.style={circle, draw=yellow!60, fill=yellow!5, very thick, minimum size=7mm},
                          roundnode/.style={circle, draw=red!60, fill=red!5, very thick, minimum size=7mm},
                          squarednode/.style={rectangle, draw=red!60, fill=red!5, very thick, minimum size=5mm},
                      ]
                      %               %Nodes
                      \node[roundnode]      (plus)                            {$+$};
                      \node[squarednode]    (y_1)     [below left=of plus]    {$y_1$};
                      \node[roundnode_yellow]      (sub)     [below right=of plus]   {$-$};
                      \node[squarednode]    (y_2)     [below left=of sub]     {$y_2$};
                      \node[squarednode]    (y_3)     [below right=of sub]    {$y_3$};


                      %   %Lines
                      \draw[->] (plus.south) -- (y_1.north);
                      \draw[->] (plus.south) -- (sub.north);
                      \draw[->] (sub.south) -- (y_2.north);
                      \draw[->] (sub.south) -- (y_3.north);
                  \end{tikzpicture}%
              \end{adjustbox}
              \qquad
              \begin{adjustbox}{width=0.25\textwidth, keepaspectratio}
                  \begin{tikzpicture}[
                          roundnode_red/.style={circle, draw=red!60, fill=red!5, very thick, minimum size=7mm},
                          roundnode_yellow/.style={circle, draw=yellow!60, fill=yellow!5, very thick, minimum size=7mm},
                          roundnode/.style={circle, draw=red!60, fill=red!5, very thick, minimum size=7mm},
                          squarednode/.style={rectangle, draw=red!60, fill=red!5, very thick, minimum size=5mm},
                      ]
                      %Nodes
                      \node[roundnode]      (plus)                            {$+$};
                      \node[squarednode]    (y_1)     [below left=of plus]    {$y_1$};
                      \node[roundnode_yellow]      (plus_2)     [below right=of plus]   {$+$};
                      \node[squarednode]    (y_2)     [below left=of plus_2]     {$y_2$};
                      \node[squarednode]    (y_3)     [below right=of plus_2]    {$y_3$};


                      %   %Lines
                      \draw[->] (plus.south) -- (y_1.north);
                      \draw[->] (plus.south) -- (plus_2.north);
                      \draw[->] (plus_2.south) -- (y_2.north);
                      \draw[->] (plus_2.south) -- (y_3.north);
                  \end{tikzpicture}%
              \end{adjustbox}
          \end{center}

    \item Elimina el nodo, colocando en su lugar su primer hijo:

          \begin{center}
              \begin{adjustbox}{width=0.25\textwidth, keepaspectratio}
                  \begin{tikzpicture}[
                          roundnode_red/.style={circle, draw=red!60, fill=red!5, very thick, minimum size=7mm},
                          roundnode_yellow/.style={circle, draw=yellow!60, fill=yellow!5, very thick, minimum size=7mm},
                          roundnode/.style={circle, draw=red!60, fill=red!5, very thick, minimum size=7mm},
                          squarednode/.style={rectangle, draw=red!60, fill=red!5, very thick, minimum size=5mm},
                      ]
                      %Nodes
                      \node[roundnode]      (plus)                            {$+$};
                      \node[squarednode]    (y_1)     [below left=of plus]    {$y_1$};
                      \node[roundnode_yellow]      (sub)     [below right=of plus]   {$-$};
                      \node[squarednode]    (y_2)     [below left=of sub]     {$y_2$};
                      \node[squarednode]    (y_3)     [below right=of sub]    {$y_3$};


                      %   %Lines
                      \draw[->] (plus.south) -- (y_1.north);
                      \draw[->] (plus.south) -- (sub.north);
                      \draw[->] (sub.south) -- (y_2.north);
                      \draw[->] (sub.south) -- (y_3.north);
                  \end{tikzpicture}%
              \end{adjustbox}
              \qquad
              \begin{adjustbox}{width=0.25\textwidth, keepaspectratio}
                  \begin{tikzpicture}[
                          roundnode_red/.style={circle, draw=red!60, fill=red!5, very thick, minimum size=7mm},
                          roundnode_yellow/.style={circle, draw=yellow!60, fill=yellow!5, very thick, minimum size=7mm},
                          roundnode/.style={circle, draw=red!60, fill=red!5, very thick, minimum size=7mm},
                          squarednode/.style={rectangle, draw=red!60, fill=red!5, very thick, minimum size=5mm},
                      ]
                      %Nodes
                      \node[roundnode]      (plus)                            {$+$};
                      \node[squarednode]    (y_1)     [below left=of plus]    {$y_1$};
                      \node[squarednode]    (y_2)     [below right=of plus]     {$y_2$};


                      %   %Lines
                      \draw[->] (plus.south) -- (y_1.north);
                      \draw[->] (plus.south) -- (y_2.north);
                  \end{tikzpicture}%
              \end{adjustbox}
          \end{center}

    \item Cambia el nodo por un nuevo representando una operación aletoria colocando como hijos nuevos árboles de expresiones aleatorias y colocando como último hijo el nodo original seleccionado:

          \begin{center}
              \begin{adjustbox}{width=0.25\textwidth, keepaspectratio}
                  \begin{tikzpicture}[
                          roundnode_red/.style={circle, draw=red!60, fill=red!5, very thick, minimum size=7mm},
                          roundnode_yellow/.style={circle, draw=yellow!60, fill=yellow!5, very thick, minimum size=7mm},
                          roundnode/.style={circle, draw=red!60, fill=red!5, very thick, minimum size=7mm},
                          squarednode/.style={rectangle, draw=red!60, fill=red!5, very thick, minimum size=5mm},
                      ]
                      %Nodes
                      \node[roundnode]      (plus)                            {$+$};
                      \node[squarednode]    (y_1)     [below left=of plus]    {$y_1$};
                      \node[roundnode_yellow]      (sub)     [below right=of plus]   {$-$};
                      \node[squarednode]    (y_2)     [below left=of sub]     {$y_2$};
                      \node[squarednode]    (y_3)     [below right=of sub]    {$y_3$};


                      %   %Lines
                      \draw[->] (plus.south) -- (y_1.north);
                      \draw[->] (plus.south) -- (sub.north);
                      \draw[->] (sub.south) -- (y_2.north);
                      \draw[->] (sub.south) -- (y_3.north);
                  \end{tikzpicture}%
              \end{adjustbox}
              \qquad
              \begin{adjustbox}{width=0.25\textwidth, keepaspectratio}
                  \begin{tikzpicture}[
                          roundnode_red/.style={circle, draw=red!60, fill=red!5, very thick, minimum size=7mm},
                          roundnode_yellow/.style={circle, draw=yellow!60, fill=yellow!5, very thick, minimum size=7mm},
                          roundnode/.style={circle, draw=red!60, fill=red!5, very thick, minimum size=7mm},
                          squarednode/.style={rectangle, draw=red!60, fill=red!5, very thick, minimum size=5mm},
                      ]
                      %Nodes
                      \node[roundnode]      (plus)                            {$+$};
                      \node[squarednode]    (y_1)     [below left=of plus]    {$y_1$};
                      \node[roundnode]    (star)  [below right=of plus]   {$*$};
                      \node[squarednode]  (y_4)   [below left=of star]        {$y_4$};
                      \node[roundnode_yellow]      (sub)     [below right=of star]   {$-$};
                      \node[squarednode]    (y_2)     [below left=of sub]     {$y_2$};
                      \node[squarednode]    (y_3)     [below right=of sub]    {$y_3$};


                      %   %Lines
                      \draw[->] (plus.south) -- (y_1.north);
                      \draw[->] (plus.south) -- (star.north);
                      \draw[->] (star.south) -- (y_4.north);
                      \draw[->] (star.south) -- (sub.north);
                      \draw[->] (sub.south) -- (y_2.north);
                      \draw[->] (sub.south) -- (y_3.north);
                  \end{tikzpicture}%
              \end{adjustbox}
          \end{center}

\end{itemize}

Si el nodo representa una variable:

\begin{itemize}
    \item Cambia la variable por otra permitida dentro de la ecuación:

          \begin{center}
              \begin{adjustbox}{width=0.25\textwidth, keepaspectratio}
                  \begin{tikzpicture}[
                          roundnode_red/.style={circle, draw=red!60, fill=red!5, very thick, minimum size=7mm},
                          roundnode/.style={circle, draw=red!60, fill=red!5, very thick, minimum size=7mm},
                          squarednode/.style={rectangle, draw=red!60, fill=red!5, very thick, minimum size=5mm},
                          squarednode_yellow/.style={rectangle, draw=yellow!60, fill=yellow!5, very thick, minimum size=7mm},
                      ]
                      %Nodes
                      \node[roundnode]      (plus)                            {$+$};
                      \node[squarednode]    (y_1)     [below left=of plus]    {$y_1$};
                      \node[roundnode]      (sub)     [below right=of plus]   {$-$};
                      \node[squarednode]    (y_2)     [below left=of sub]     {$y_2$};
                      \node[squarednode_yellow]    (y_3)     [below right=of sub]    {$y_3$};


                      %   %Lines
                      \draw[->] (plus.south) -- (y_1.north);
                      \draw[->] (plus.south) -- (sub.north);
                      \draw[->] (sub.south) -- (y_2.north);
                      \draw[->] (sub.south) -- (y_3.north);
                  \end{tikzpicture}%
              \end{adjustbox}
              \qquad
              \begin{adjustbox}{width=0.25\textwidth, keepaspectratio}
                  \begin{tikzpicture}[
                          roundnode_red/.style={circle, draw=red!60, fill=red!5, very thick, minimum size=7mm},
                          roundnode/.style={circle, draw=red!60, fill=red!5, very thick, minimum size=7mm},
                          squarednode/.style={rectangle, draw=red!60, fill=red!5, very thick, minimum size=5mm},
                          squarednode_yellow/.style={rectangle, draw=yellow!60, fill=yellow!5, very thick, minimum size=7mm},
                      ]
                      %Nodes
                      \node[roundnode]      (plus)                            {$+$};
                      \node[squarednode]    (y_1)     [below left=of plus]    {$y_1$};
                      \node[roundnode]      (sub)     [below right=of plus]   {$-$};
                      \node[squarednode]    (y_2)     [below left=of sub]     {$y_2$};
                      \node[squarednode_yellow]    (y_1_2)     [below right=of sub]    {$y_1$};


                      %   %Lines
                      \draw[->] (plus.south) -- (y_1.north);
                      \draw[->] (plus.south) -- (sub.north);
                      \draw[->] (sub.south) -- (y_2.north);
                      \draw[->] (sub.south) -- (y_1_2.north);
                  \end{tikzpicture}%
              \end{adjustbox}
          \end{center}

    \item Cambiar la variable por un nodo que represente una operación aleatoria colocando como hijos variables que puedan existir en la ecuación y colocando como primer hijo la variable original seleccionada:

          \begin{center}
              \begin{adjustbox}{width=0.25\textwidth, keepaspectratio}
                  \begin{tikzpicture}[
                          roundnode_red/.style={circle, draw=red!60, fill=red!5, very thick, minimum size=7mm},
                          roundnode/.style={circle, draw=red!60, fill=red!5, very thick, minimum size=7mm},
                          squarednode/.style={rectangle, draw=red!60, fill=red!5, very thick, minimum size=5mm},
                          squarednode_yellow/.style={rectangle, draw=yellow!60, fill=yellow!5, very thick, minimum size=7mm},
                      ]
                      %Nodes
                      \node[roundnode]      (plus)                            {$+$};
                      \node[squarednode]    (y_1)     [below left=of plus]    {$y_1$};
                      \node[roundnode]      (sub)     [below right=of plus]   {$-$};
                      \node[squarednode]    (y_2)     [below left=of sub]     {$y_2$};
                      \node[squarednode_yellow]    (y_3)     [below right=of sub]    {$y_3$};


                      %   %Lines
                      \draw[->] (plus.south) -- (y_1.north);
                      \draw[->] (plus.south) -- (sub.north);
                      \draw[->] (sub.south) -- (y_2.north);
                      \draw[->] (sub.south) -- (y_3.north);
                  \end{tikzpicture}%
              \end{adjustbox}
              \qquad
              \begin{adjustbox}{width=0.25\textwidth, keepaspectratio}
                  \begin{tikzpicture}[
                          roundnode_red/.style={circle, draw=red!60, fill=red!5, very thick, minimum size=7mm},
                          roundnode/.style={circle, draw=red!60, fill=red!5, very thick, minimum size=7mm},
                          squarednode/.style={rectangle, draw=red!60, fill=red!5, very thick, minimum size=5mm},
                          squarednode_yellow/.style={rectangle, draw=yellow!60, fill=yellow!5, very thick, minimum size=7mm},
                      ]
                      %Nodes
                      \node[roundnode]      (plus)                            {$+$};
                      \node[squarednode]    (y_1)     [below left=of plus]    {$y_1$};
                      \node[roundnode]    (sub)  [below right=of plus]   {$-$};
                      \node[squarednode]  (y_2)   [below left=of sub]        {$y_2$};
                      \node[roundnode]      (star)     [below right=of sub]   {$*$};
                      \node[squarednode_yellow]    (y_3)     [below left=of star]    {$y_3$};
                      \node[squarednode]    (y_2_2)     [below right=of star]     {$y_2$};


                      %   %Lines
                      \draw[->] (plus.south) -- (y_1.north);
                      \draw[->] (plus.south) -- (sub.north);
                      \draw[->] (sub.south) -- (y_2.north);
                      \draw[->] (sub.south) -- (star.north);
                      \draw[->] (star.south) -- (y_3.north);
                      \draw[->] (star.south) -- (y_2_2.north);
                  \end{tikzpicture}%
              \end{adjustbox}
          \end{center}

\end{itemize}

Por ejemplo si se tiene el árbol

\begin{center}
    \begin{adjustbox}{width=0.5\textwidth, keepaspectratio}
        \begin{tikzpicture}[
                roundnode/.style={circle, draw=red!60, fill=red!5, very thick, minimum size=7mm},
                squarednode/.style={rectangle, draw=red!60, fill=red!5, very thick, minimum size=5mm},
                roundnode_yellow/.style={circle, draw=yellow!60, fill=yellow!5, very thick, minimum size=7mm},
            ]
            % Nodes
            \node[roundnode]        (system)                            {$SYSTEM$};

            \node[roundnode]        (plus_S)     [below left=of system]        {$+$};
            \node[roundnode]        (star_S_1)    [below left=of plus_S]    {$*$};
            \node[squarednode]      (alpha_star_S_1)      [below left=of star_S_1]    {$a_1$};
            \node[roundnode]        (neg_star_S_1)    [below=of star_S_1]    {$-$};
            \node[roundnode]        (star_S_2)    [below=of neg_star_S_1]    {$*$};
            \node[squarednode]      (S_star_S)       [below left=of star_S_2]   {$S$};
            \node[squarednode]      (I_star_S)      [below=of star_S_2]   {$I$};

            \node[roundnode]        (plus_I)     [below=of system]        {$+$};
            \node[roundnode]        (star_I_1)    [below left=of plus_I]    {$*$};
            \node[squarednode]      (alpha_star_I_1)      [below left=1cm and 0.5cm of star_I_1]    {$a_2$};
            \node[roundnode]        (star_I_2)    [below=of star_I_1]    {$*$};
            \node[squarednode]      (S_star_I)       [below left=1cm and 0.5cm of star_I_2]   {$S$};
            \node[squarednode]      (I_star_I_1)      [below=of star_I_2]   {$I$};

            \node[roundnode_yellow]        (star_I_3)    [below right=of plus_I]    {$*$};
            \node[squarednode]      (beta_star_I_1)      [below=of star_I_3]    {$a_3$};
            \node[roundnode]        (neg_star_I_1)    [below right=1cm and 0.5cm of star_I_3]    {$-$};
            \node[squarednode]      (I_star_I_2)      [below=of neg_star_I_1]   {$I$};

            \node[roundnode]        (plus_R)     [below right=of system]        {$+$};
            \node[roundnode]        (star_R_1)    [below right=of plus_R]    {$*$};
            \node[squarednode]      (beta_star_R_1)      [below=of star_R_1]    {$a_4$};
            \node[squarednode]      (I_star_R)      [below right=of star_R_1]   {$I$};

            %Lines
            \draw [->] (system.south) -- (plus_S.north);
            \draw[->] (system.south) -- (plus_I.north);
            \draw[->] (system.south) -- (plus_R.north);

            \draw[->] (plus_S.south) -- (star_S_1.north);
            \draw[->] (star_S_1.south) -- (alpha_star_S_1.north);
            \draw[->] (star_S_1.south) -- (neg_star_S_1.north);
            \draw[->] (neg_star_S_1.south) -- (star_S_2.north);
            \draw[->] (star_S_2.south) -- (S_star_S.north);
            \draw[->] (star_S_2.south) -- (I_star_S.north);

            \draw[->] (plus_I.south) -- (star_I_1.north);
            \draw[->] (plus_I.south) -- (star_I_3.north);
            \draw[->] (star_I_1.south) -- (alpha_star_I_1.north);
            \draw[->] (star_I_1.south) -- (star_I_2.north);
            \draw[->] (star_I_2.south) -- (S_star_I.north);
            \draw[->] (star_I_2.south) -- (I_star_I_1.north);

            \draw[->] (star_I_3.south) -- (beta_star_I_1.north);
            \draw[->] (star_I_3.south) -- (neg_star_I_1.north);
            \draw[->] (neg_star_I_1.south) -- (I_star_I_2.north);

            \draw[->] (plus_R.south) -- (star_R_1.north);
            \draw[->] (star_R_1.south) -- (beta_star_R_1.north);
            \draw[->] (star_R_1.south) -- (I_star_R.north);
        \end{tikzpicture}
    \end{adjustbox}
\end{center}

que representa al modelo

\begin{align*}
    S' & = - a_1 * S * I         \\
    I' & = a_2 * S * I - a_3 * I \\
    R' & = a_4 * I,
\end{align*}


se puede realizar la mutación de eliminar el segundo término de la segunda ecuación quedando como resultado el árbol:

\begin{center}
    \begin{adjustbox}{width=0.5\textwidth, keepaspectratio}
        \begin{tikzpicture}[
                roundnode/.style={circle, draw=red!60, fill=red!5, very thick, minimum size=7mm},
                squarednode/.style={rectangle, draw=red!60, fill=red!5, very thick, minimum size=5mm},
                roundnode_yellow/.style={circle, draw=yellow!60, fill=yellow!5, very thick, minimum size=7mm},
            ]
            % Nodes
            \node[roundnode]        (system)                            {$SYSTEM$};

            \node[roundnode]        (plus_S)     [below left=of system]        {$+$};
            \node[roundnode]        (star_S_1)    [below left=of plus_S]    {$*$};
            \node[squarednode]      (alpha_star_S_1)      [below left=of star_S_1]    {$a_1$};
            \node[roundnode]        (neg_star_S_1)    [below=of star_S_1]    {$-$};
            \node[roundnode]        (star_S_2)    [below=of neg_star_S_1]    {$*$};
            \node[squarednode]      (S_star_S)       [below left=of star_S_2]   {$S$};
            \node[squarednode]      (I_star_S)      [below=of star_S_2]   {$I$};

            \node[roundnode]        (plus_I)     [below=of system]        {$+$};
            \node[roundnode]        (star_I_1)    [below=of plus_I]    {$*$};
            \node[squarednode]      (alpha_star_I_1)      [below left=of star_I_1]    {$a_2$};
            \node[roundnode]        (star_I_2)    [below right=of star_I_1]    {$*$};
            \node[squarednode]      (S_star_I)       [below left=of star_I_2]   {$S$};
            \node[squarednode]      (I_star_I_1)      [below=of star_I_2]   {$I$};

            \node[roundnode]        (plus_R)     [below right=of system]        {$+$};
            \node[roundnode]        (star_R_1)    [below right=of plus_R]    {$*$};
            \node[squarednode]      (beta_star_R_1)      [below=of star_R_1]    {$a_4$};
            \node[squarednode]      (I_star_R)      [below right=of star_R_1]   {$I$};

            %Lines
            \draw [->] (system.south) -- (plus_S.north);
            \draw[->] (system.south) -- (plus_I.north);
            \draw[->] (system.south) -- (plus_R.north);

            \draw[->] (plus_S.south) -- (star_S_1.north);
            \draw[->] (star_S_1.south) -- (alpha_star_S_1.north);
            \draw[->] (star_S_1.south) -- (neg_star_S_1.north);
            \draw[->] (neg_star_S_1.south) -- (star_S_2.north);
            \draw[->] (star_S_2.south) -- (S_star_S.north);
            \draw[->] (star_S_2.south) -- (I_star_S.north);

            \draw[->] (plus_I.south) -- (star_I_1.north);
            \draw[->] (star_I_1.south) -- (alpha_star_I_1.north);
            \draw[->] (star_I_1.south) -- (star_I_2.north);
            \draw[->] (star_I_2.south) -- (S_star_I.north);
            \draw[->] (star_I_2.south) -- (I_star_I_1.north);

            \draw[->] (plus_R.south) -- (star_R_1.north);
            \draw[->] (star_R_1.south) -- (beta_star_R_1.north);
            \draw[->] (star_R_1.south) -- (I_star_R.north);
        \end{tikzpicture}
    \end{adjustbox}
\end{center}

que representa al sistema:

\begin{align*}
    S' & = - a_1 * S * I \\
    I' & = a_2 * S * I   \\
    R' & = a_4 * I.
\end{align*}

Otra de las operaciones que se deben definir con el fin de implementar un algoritmo genético es el cruzamiento, esta operación se específica en la siguiente sección.

\section{Cruzamiento}

La operación de cruzamiento selecciona dos sistemas aleatorios dentro de la población de soluciones presentes en el algoritmo genético, definamos estos sistemas como A y B sin perder generalidad. Luego se selecciona un nodo aleatorio dentro del árbol A, impidiendo que la selección tome hojas que representen parámetros. En dependencia de la altura del nodo seleccionado se escoge un nodo en B siguiendo las siguientes reglas:

\begin{itemize}
    \item Si el nodo seleccionado en A es el representante de la i-ésima ecuación en el sistema, se escoge el nodo representante de la i-ésima ecuación en el sistema B:

          \begin{center}
              \begin{adjustbox}{width=0.4\textwidth, keepaspectratio}
                  \begin{tikzpicture}[
                          roundnode/.style={circle, draw=red!60, fill=red!5, very thick, minimum size=7mm},
                          squarednode/.style={rectangle, draw=red!60, fill=red!5, very thick, minimum size=5mm},
                          roundnode_yellow/.style={circle, draw=yellow!60, fill=yellow!5, very thick, minimum size=7mm},
                      ]
                      % Nodes
                      \node[roundnode]        (system)                            {$A$};

                      \node[roundnode]        (plus_1)     [below left=of system]        {$+$};
                      \node[roundnode]        (star_1_1)    [below=of plus_1]    {$*$};
                      \node[squarednode]      (a_1)      [below left=of star_1_1]    {$a_1$};
                      \node[roundnode]        (star_1_2)    [below=of star_1_1]    {$*$};
                      \node[squarednode]      (S_star_1_2)       [below left=of star_1_2]   {$S$};
                      \node[squarednode]      (I_star_1_2)      [below=of star_1_2]   {$I$};

                      \node[roundnode_yellow]        (plus_2)     [below right=of system]        {$+$};
                      \node[roundnode]        (star_2_1)    [below=of plus_2]    {$*$};
                      \node[squarednode]      (a_2)      [below=of star_2_1]    {$a_2$};
                      \node[squarednode]      (I_star_2_1)      [below right=of star_2_1]   {$I$};

                      %Lines
                      \draw [->] (system.south) -- (plus_1.north);
                      \draw[->] (system.south) -- (plus_2.north);
                      \draw[->] (plus_1.south) -- (star_1_1.north);
                      \draw[->] (plus_2.south) -- (star_2_1.north);
                      \draw[->] (star_1_1.south) -- (a_1.north);
                      \draw[->] (star_1_1.south) -- (star_1_2.north);
                      \draw[->] (star_1_2.south) -- (S_star_1_2.north);
                      \draw[->] (star_1_2.south) -- (I_star_1_2.north);
                      \draw[->] (star_2_1.south) -- (a_2.north);
                      \draw[->] (star_2_1.south) -- (I_star_2_1.north);
                  \end{tikzpicture}
              \end{adjustbox}
              \qquad
              \begin{adjustbox}{width=0.4\textwidth, keepaspectratio}
                  \begin{tikzpicture}[
                          roundnode/.style={circle, draw=red!60, fill=red!5, very thick, minimum size=7mm},
                          squarednode/.style={rectangle, draw=red!60, fill=red!5, very thick, minimum size=5mm},
                          roundnode_yellow/.style={circle, draw=yellow!60, fill=yellow!5, very thick, minimum size=7mm},
                      ]
                      % Nodes
                      \node[roundnode]        (system)                            {$B$};

                      \node[roundnode]        (plus_1)     [below left=of system]        {$+$};
                      \node[roundnode]        (star_1_1)    [below=of plus_1]    {$*$};
                      \node[squarednode]      (a_3)      [below left=of star_1_1]    {$a_3$};
                      \node[roundnode]        (plus_1_2)    [below=of star_1_1]    {$+$};
                      \node[squarednode]      (I_plus_1_2)       [below left=of plus_1_2]   {$I$};
                      \node[squarednode]      (S_plus_1_2)      [below=of plus_1_2]   {$S$};

                      \node[roundnode_yellow]        (plus_2)     [below right=of system]        {$+$};
                      \node[roundnode]        (star_2_1)    [below=of plus_2]    {$*$};
                      \node[squarednode]      (a_4)      [below=of star_2_1]    {$a_4$};
                      \node[roundnode]        (sub_2_1)    [below right=of star_2_1]    {$-$};
                      \node[squarednode]      (S_sub_2_1)       [below left=of sub_2_1]   {$S$};
                      \node[squarednode]      (I_sub_2_1)      [below=of sub_2_1]   {$I$};

                      %Lines
                      \draw [->] (system.south) -- (plus_1.north);
                      \draw[->] (system.south) -- (plus_2.north);
                      \draw[->] (plus_1.south) -- (star_1_1.north);
                      \draw[->] (plus_2.south) -- (star_2_1.north);
                      \draw[->] (star_1_1.south) -- (a_3.north);
                      \draw[->] (star_1_1.south) -- (plus_1_2.north);
                      \draw[->] (plus_1_2.south) -- (I_plus_1_2.north);
                      \draw[->] (plus_1_2.south) -- (S_plus_1_2.north);
                      \draw[->] (star_2_1.south) -- (a_4.north);
                      \draw[->] (star_2_1.south) -- (sub_2_1.north);
                      \draw[->] (sub_2_1.south) -- (S_sub_2_1.north);
                      \draw[->] (sub_2_1.south) -- (I_sub_2_1.north);
                  \end{tikzpicture}
              \end{adjustbox}
          \end{center}

    \item Si se selecciona en A un nodo representante de un término en la ecuación i-ésima ecuación, se escoge un nodo representante de un término en la i-ésima ecuación en el sistema B:

          \begin{center}
              \begin{adjustbox}{width=0.4\textwidth, keepaspectratio}
                  \begin{tikzpicture}[
                          roundnode/.style={circle, draw=red!60, fill=red!5, very thick, minimum size=7mm},
                          squarednode/.style={rectangle, draw=red!60, fill=red!5, very thick, minimum size=5mm},
                          roundnode_yellow/.style={circle, draw=yellow!60, fill=yellow!5, very thick, minimum size=7mm},
                      ]
                      % Nodes
                      \node[roundnode]        (system)                            {$A$};

                      \node[roundnode]        (plus_1)     [below left=of system]        {$+$};
                      \node[roundnode_yellow]        (star_1_1)    [below=of plus_1]    {$*$};
                      \node[squarednode]      (a_1)      [below left=of star_1_1]    {$a_1$};
                      \node[roundnode]        (star_1_2)    [below=of star_1_1]    {$*$};
                      \node[squarednode]      (S_star_1_2)       [below left=of star_1_2]   {$S$};
                      \node[squarednode]      (I_star_1_2)      [below=of star_1_2]   {$I$};

                      \node[roundnode]        (plus_2)     [below right=of system]        {$+$};
                      \node[roundnode]        (star_2_1)    [below=of plus_2]    {$*$};
                      \node[squarednode]      (a_2)      [below=of star_2_1]    {$a_2$};
                      \node[squarednode]      (I_star_2_1)      [below right=of star_2_1]   {$I$};

                      %Lines
                      \draw [->] (system.south) -- (plus_1.north);
                      \draw[->] (system.south) -- (plus_2.north);
                      \draw[->] (plus_1.south) -- (star_1_1.north);
                      \draw[->] (plus_2.south) -- (star_2_1.north);
                      \draw[->] (star_1_1.south) -- (a_1.north);
                      \draw[->] (star_1_1.south) -- (star_1_2.north);
                      \draw[->] (star_1_2.south) -- (S_star_1_2.north);
                      \draw[->] (star_1_2.south) -- (I_star_1_2.north);
                      \draw[->] (star_2_1.south) -- (a_2.north);
                      \draw[->] (star_2_1.south) -- (I_star_2_1.north);
                  \end{tikzpicture}
              \end{adjustbox}
              \qquad
              \begin{adjustbox}{width=0.4\textwidth, keepaspectratio}
                  \begin{tikzpicture}[
                          roundnode/.style={circle, draw=red!60, fill=red!5, very thick, minimum size=7mm},
                          squarednode/.style={rectangle, draw=red!60, fill=red!5, very thick, minimum size=5mm},
                          roundnode_yellow/.style={circle, draw=yellow!60, fill=yellow!5, very thick, minimum size=7mm},
                      ]
                      % Nodes
                      \node[roundnode]        (system)                            {$B$};

                      \node[roundnode]        (plus_1)     [below left=of system]        {$+$};
                      \node[roundnode_yellow]        (star_1_1)    [below=of plus_1]    {$*$};
                      \node[squarednode]      (a_3)      [below left=of star_1_1]    {$a_3$};
                      \node[roundnode]        (plus_1_2)    [below=of star_1_1]    {$+$};
                      \node[squarednode]      (I_plus_1_2)       [below left=of plus_1_2]   {$I$};
                      \node[squarednode]      (S_plus_1_2)      [below=of plus_1_2]   {$S$};

                      \node[roundnode]        (plus_2)     [below right=of system]        {$+$};
                      \node[roundnode]        (star_2_1)    [below=of plus_2]    {$*$};
                      \node[squarednode]      (a_4)      [below=of star_2_1]    {$a_4$};
                      \node[roundnode]        (sub_2_1)    [below right=of star_2_1]    {$-$};
                      \node[squarednode]      (S_sub_2_1)       [below left=of sub_2_1]   {$S$};
                      \node[squarednode]      (I_sub_2_1)      [below=of sub_2_1]   {$I$};

                      %Lines
                      \draw [->] (system.south) -- (plus_1.north);
                      \draw[->] (system.south) -- (plus_2.north);
                      \draw[->] (plus_1.south) -- (star_1_1.north);
                      \draw[->] (plus_2.south) -- (star_2_1.north);
                      \draw[->] (star_1_1.south) -- (a_3.north);
                      \draw[->] (star_1_1.south) -- (plus_1_2.north);
                      \draw[->] (plus_1_2.south) -- (I_plus_1_2.north);
                      \draw[->] (plus_1_2.south) -- (S_plus_1_2.north);
                      \draw[->] (star_2_1.south) -- (a_4.north);
                      \draw[->] (star_2_1.south) -- (sub_2_1.north);
                      \draw[->] (sub_2_1.south) -- (S_sub_2_1.north);
                      \draw[->] (sub_2_1.south) -- (I_sub_2_1.north);
                  \end{tikzpicture}
              \end{adjustbox}
          \end{center}


    \item Si el nodo seleccionado en A es el representante de una operación o una variable perteneciente a la i-ésima ecuación, se selecciona un nodo representante de una operación o una variable que pertenezca a algún término en la i-ésima ecuación en el sistema B:


          \begin{center}
              \begin{adjustbox}{width=0.4\textwidth, keepaspectratio}
                  \begin{tikzpicture}[
                          roundnode/.style={circle, draw=red!60, fill=red!5, very thick, minimum size=7mm},
                          squarednode/.style={rectangle, draw=red!60, fill=red!5, very thick, minimum size=5mm},
                          roundnode_yellow/.style={circle, draw=yellow!60, fill=yellow!5, very thick, minimum size=7mm},
                      ]
                      % Nodes
                      \node[roundnode]        (system)                            {$A$};

                      \node[roundnode]        (plus_1)     [below left=of system]        {$+$};
                      \node[roundnode]        (star_1_1)    [below=of plus_1]    {$*$};
                      \node[squarednode]      (a_1)      [below left=of star_1_1]    {$a_1$};
                      \node[roundnode]        (star_1_2)    [below=of star_1_1]    {$*$};
                      \node[squarednode]      (S_star_1_2)       [below left=of star_1_2]   {$S$};
                      \node[squarednode]      (I_star_1_2)      [below=of star_1_2]   {$I$};

                      \node[roundnode]        (plus_2)     [below right=of system]        {$+$};
                      \node[roundnode]        (star_2_1)    [below=of plus_2]    {$*$};
                      \node[squarednode]      (a_2)      [below=of star_2_1]    {$a_2$};
                      \node[roundnode_yellow]      (I_star_2_1)      [below right=of star_2_1]   {$I$};

                      %Lines
                      \draw [->] (system.south) -- (plus_1.north);
                      \draw[->] (system.south) -- (plus_2.north);
                      \draw[->] (plus_1.south) -- (star_1_1.north);
                      \draw[->] (plus_2.south) -- (star_2_1.north);
                      \draw[->] (star_1_1.south) -- (a_1.north);
                      \draw[->] (star_1_1.south) -- (star_1_2.north);
                      \draw[->] (star_1_2.south) -- (S_star_1_2.north);
                      \draw[->] (star_1_2.south) -- (I_star_1_2.north);
                      \draw[->] (star_2_1.south) -- (a_2.north);
                      \draw[->] (star_2_1.south) -- (I_star_2_1.north);
                  \end{tikzpicture}
              \end{adjustbox}
              \qquad
              \begin{adjustbox}{width=0.4\textwidth, keepaspectratio}
                  \begin{tikzpicture}[
                          roundnode/.style={circle, draw=red!60, fill=red!5, very thick, minimum size=7mm},
                          squarednode/.style={rectangle, draw=red!60, fill=red!5, very thick, minimum size=5mm},
                          roundnode_yellow/.style={circle, draw=yellow!60, fill=yellow!5, very thick, minimum size=7mm},
                      ]
                      % Nodes
                      \node[roundnode]        (system)                            {$B$};

                      \node[roundnode]        (plus_1)     [below left=of system]        {$+$};
                      \node[roundnode]        (star_1_1)    [below=of plus_1]    {$*$};
                      \node[squarednode]      (a_3)      [below left=of star_1_1]    {$a_3$};
                      \node[roundnode]        (plus_1_2)    [below=of star_1_1]    {$+$};
                      \node[squarednode]      (I_plus_1_2)       [below left=of plus_1_2]   {$I$};
                      \node[squarednode]      (S_plus_1_2)      [below=of plus_1_2]   {$S$};

                      \node[roundnode]        (plus_2)     [below right=of system]        {$+$};
                      \node[roundnode]        (star_2_1)    [below=of plus_2]    {$*$};
                      \node[squarednode]      (a_4)      [below=of star_2_1]    {$a_4$};
                      \node[roundnode_yellow]        (sub_2_1)    [below right=of star_2_1]    {$-$};
                      \node[squarednode]      (S_sub_2_1)       [below left=of sub_2_1]   {$S$};
                      \node[squarednode]      (I_sub_2_1)      [below=of sub_2_1]   {$I$};

                      %Lines
                      \draw [->] (system.south) -- (plus_1.north);
                      \draw[->] (system.south) -- (plus_2.north);
                      \draw[->] (plus_1.south) -- (star_1_1.north);
                      \draw[->] (plus_2.south) -- (star_2_1.north);
                      \draw[->] (star_1_1.south) -- (a_3.north);
                      \draw[->] (star_1_1.south) -- (plus_1_2.north);
                      \draw[->] (plus_1_2.south) -- (I_plus_1_2.north);
                      \draw[->] (plus_1_2.south) -- (S_plus_1_2.north);
                      \draw[->] (star_2_1.south) -- (a_4.north);
                      \draw[->] (star_2_1.south) -- (sub_2_1.north);
                      \draw[->] (sub_2_1.south) -- (S_sub_2_1.north);
                      \draw[->] (sub_2_1.south) -- (I_sub_2_1.north);
                  \end{tikzpicture}
              \end{adjustbox}
          \end{center}

\end{itemize}

Luego de que se selecciona un nodo en el sistema A y un nodo en el sistema B, se sustituye el subárbol del nodo del árbol B en el A resultado un nuevo sistema en lo que fue el sistema A.

Por ejemplo si se tienen los árboles:

\begin{center}
    \begin{adjustbox}{width=0.5\textwidth, keepaspectratio}
        \begin{tikzpicture}[
                roundnode/.style={circle, draw=red!60, fill=red!5, very thick, minimum size=7mm},
                squarednode/.style={rectangle, draw=red!60, fill=red!5, very thick, minimum size=5mm},
                roundnode_yellow/.style={circle, draw=yellow!60, fill=yellow!5, very thick, minimum size=7mm},
            ]
            % Nodes
            \node[roundnode]        (system)                            {$A$};

            \node[roundnode]        (plus_S)     [below left=of system]        {$+$};
            \node[roundnode]        (star_S_1)    [below left=of plus_S]    {$*$};
            \node[squarednode]      (alpha_star_S_1)      [below left=of star_S_1]    {$a_1$};
            \node[roundnode]        (neg_star_S_1)    [below=of star_S_1]    {$-$};
            \node[roundnode]        (star_S_2)    [below=of neg_star_S_1]    {$*$};
            \node[squarednode]      (S_star_S)       [below left=of star_S_2]   {$S$};
            \node[squarednode]      (I_star_S)      [below=of star_S_2]   {$I$};

            \node[roundnode_yellow]        (plus_I)     [below=of system]        {$+$};
            \node[roundnode]        (star_I_1)    [below left=1cm and 0.5cm of plus_I]    {$*$};
            \node[squarednode]      (alpha_star_I_1)      [below left=1cm and 0.5cm of star_I_1]    {$a_2$};
            \node[roundnode]        (star_I_2)    [below=of star_I_1]    {$*$};
            \node[squarednode]      (S_star_I)       [below left=1cm and 0.5cm of star_I_2]   {$S$};
            \node[squarednode]      (I_star_I_1)      [below=of star_I_2]   {$I$};

            \node[roundnode]        (star_I_3)    [below right=1cm and 0.5cm of plus_I]    {$*$};
            \node[squarednode]      (beta_star_I_1)      [below=of star_I_3]    {$a_3$};
            \node[roundnode]        (neg_star_I_1)    [below right=1cm and 0.5cm of star_I_3]    {$-$};
            \node[squarednode]      (I_star_I_2)      [below=of neg_star_I_1]   {$I$};

            \node[roundnode]        (plus_R)     [below right=of system]        {$+$};
            \node[roundnode]        (star_R_1)    [below right=of plus_R]    {$*$};
            \node[squarednode]      (beta_star_R_1)      [below=of star_R_1]    {$a_4$};
            \node[squarednode]      (I_star_R)      [below right=of star_R_1]   {$I$};

            %Lines
            \draw [->] (system.south) -- (plus_S.north);
            \draw[->] (system.south) -- (plus_I.north);
            \draw[->] (system.south) -- (plus_R.north);

            \draw[->] (plus_S.south) -- (star_S_1.north);
            \draw[->] (star_S_1.south) -- (alpha_star_S_1.north);
            \draw[->] (star_S_1.south) -- (neg_star_S_1.north);
            \draw[->] (neg_star_S_1.south) -- (star_S_2.north);
            \draw[->] (star_S_2.south) -- (S_star_S.north);
            \draw[->] (star_S_2.south) -- (I_star_S.north);

            \draw[->] (plus_I.south) -- (star_I_1.north);
            \draw[->] (plus_I.south) -- (star_I_3.north);
            \draw[->] (star_I_1.south) -- (alpha_star_I_1.north);
            \draw[->] (star_I_1.south) -- (star_I_2.north);
            \draw[->] (star_I_2.south) -- (S_star_I.north);
            \draw[->] (star_I_2.south) -- (I_star_I_1.north);

            \draw[->] (star_I_3.south) -- (beta_star_I_1.north);
            \draw[->] (star_I_3.south) -- (neg_star_I_1.north);
            \draw[->] (neg_star_I_1.south) -- (I_star_I_2.north);

            \draw[->] (plus_R.south) -- (star_R_1.north);
            \draw[->] (star_R_1.south) -- (beta_star_R_1.north);
            \draw[->] (star_R_1.south) -- (I_star_R.north);
        \end{tikzpicture}
    \end{adjustbox}
\end{center}

que representa el sistema:

\begin{align*}
    S' & = - a_1 * S * I         \\
    I' & = a_2 * S * I - a_3 * I \\
    R' & = a_4 * I
\end{align*}

y

\begin{center}
    \begin{adjustbox}{width=0.5\textwidth, keepaspectratio}
        \begin{tikzpicture}[
                roundnode/.style={circle, draw=red!60, fill=red!5, very thick, minimum size=7mm},
                squarednode/.style={rectangle, draw=red!60, fill=red!5, very thick, minimum size=5mm},
                roundnode_yellow/.style={circle, draw=yellow!60, fill=yellow!5, very thick, minimum size=7mm},
            ]
            % Nodes
            \node[roundnode]        (system)                            {$B$};

            \node[roundnode]        (plus_S)     [below left=of system]        {$+$};
            \node[roundnode]        (star_S_1)    [below left=of plus_S]    {$*$};
            \node[squarednode]      (alpha_star_S_1)      [below left=of star_S_1]    {$b_1$};
            \node[roundnode]        (add_S_2)    [below=of star_S_1]    {$+$};
            \node[squarednode]      (S_star_S)       [below left=of add_S_2]   {$S$};
            \node[squarednode]      (I_star_S)      [below=of add_S_2]   {$I$};

            \node[roundnode_yellow]        (plus_I)     [below=of system]        {$+$};
            \node[roundnode]        (star_I_1)    [below=of plus_I]    {$*$};
            \node[squarednode]      (alpha_star_I_1)      [below left=of star_I_1]    {$b_2$};
            \node[roundnode]        (star_I_2)    [below right=of star_I_1]    {$*$};
            \node[squarednode]      (S_star_I)       [below left=of star_I_2]   {$S$};
            \node[squarednode]      (I_star_I_1)      [below=of star_I_2]   {$I$};

            \node[roundnode]        (plus_R)     [below right=of system]        {$+$};
            \node[roundnode]        (star_R_1)    [below right=of plus_R]    {$*$};
            \node[squarednode]      (beta_star_R_1)      [below=of star_R_1]    {$b_3$};
            \node[squarednode]      (S_star_R)      [below right=of star_R_1]   {$S$};

            %Lines
            \draw [->] (system.south) -- (plus_S.north);
            \draw[->] (system.south) -- (plus_I.north);
            \draw[->] (system.south) -- (plus_R.north);

            \draw[->] (plus_S.south) -- (star_S_1.north);
            \draw[->] (star_S_1.south) -- (alpha_star_S_1.north);
            \draw[->] (star_S_1.south) -- (add_S_2.north);
            \draw[->] (add_S_2.south) -- (S_star_S.north);
            \draw[->] (add_S_2.south) -- (I_star_S.north);

            \draw[->] (plus_I.south) -- (star_I_1.north);
            \draw[->] (star_I_1.south) -- (alpha_star_I_1.north);
            \draw[->] (star_I_1.south) -- (star_I_2.north);
            \draw[->] (star_I_2.south) -- (S_star_I.north);
            \draw[->] (star_I_2.south) -- (I_star_I_1.north);

            \draw[->] (plus_R.south) -- (star_R_1.north);
            \draw[->] (star_R_1.south) -- (beta_star_R_1.north);
            \draw[->] (star_R_1.south) -- (S_star_R.north);
        \end{tikzpicture}
    \end{adjustbox}
\end{center}

que representa al sistema:

\begin{align*}
    S' & = b_1 * (S + I) \\
    I' & = b_2 * S * I   \\
    R' & = b_3 * S.
\end{align*}

Si como resultado de la operación de cruzamiento los nodos seleccionados son los resaltados en color amarillo, entonces el árbol resultante sería:


\begin{center}
    \begin{adjustbox}{width=0.5\textwidth, keepaspectratio}
        \begin{tikzpicture}[
                roundnode/.style={circle, draw=red!60, fill=red!5, very thick, minimum size=7mm},
                squarednode/.style={rectangle, draw=red!60, fill=red!5, very thick, minimum size=5mm},
                roundnode_yellow/.style={circle, draw=yellow!60, fill=yellow!5, very thick, minimum size=7mm},
            ]
            % Nodes
            \node[roundnode]        (system)                            {$A$};

            \node[roundnode]        (plus_S)     [below left=of system]        {$+$};
            \node[roundnode]        (star_S_1)    [below left=of plus_S]    {$*$};
            \node[squarednode]      (alpha_star_S_1)      [below left=of star_S_1]    {$a_1$};
            \node[roundnode]        (neg_star_S_1)    [below=of star_S_1]    {$-$};
            \node[roundnode]        (star_S_2)    [below=of neg_star_S_1]    {$*$};
            \node[squarednode]      (S_star_S)       [below left=of star_S_2]   {$S$};
            \node[squarednode]      (I_star_S)      [below=of star_S_2]   {$I$};

            \node[roundnode_yellow]        (plus_I)     [below=of system]        {$+$};
            \node[roundnode]        (star_I_1)    [below=of plus_I]    {$*$};
            \node[squarednode]      (alpha_star_I_1)      [below left=of star_I_1]    {$b_2$};
            \node[roundnode]        (star_I_2)    [below right=of star_I_1]    {$*$};
            \node[squarednode]      (S_star_I)       [below left=of star_I_2]   {$S$};
            \node[squarednode]      (I_star_I_1)      [below=of star_I_2]   {$I$};

            \node[roundnode]        (plus_R)     [below right=of system]        {$+$};
            \node[roundnode]        (star_R_1)    [below right=of plus_R]    {$*$};
            \node[squarednode]      (beta_star_R_1)      [below=of star_R_1]    {$a_4$};
            \node[squarednode]      (I_star_R)      [below right=of star_R_1]   {$I$};

            %Lines
            \draw [->] (system.south) -- (plus_S.north);
            \draw[->] (system.south) -- (plus_I.north);
            \draw[->] (system.south) -- (plus_R.north);

            \draw[->] (plus_S.south) -- (star_S_1.north);
            \draw[->] (star_S_1.south) -- (alpha_star_S_1.north);
            \draw[->] (star_S_1.south) -- (neg_star_S_1.north);
            \draw[->] (neg_star_S_1.south) -- (star_S_2.north);
            \draw[->] (star_S_2.south) -- (S_star_S.north);
            \draw[->] (star_S_2.south) -- (I_star_S.north);

            \draw[->] (plus_I.south) -- (star_I_1.north);
            \draw[->] (star_I_1.south) -- (alpha_star_I_1.north);
            \draw[->] (star_I_1.south) -- (star_I_2.north);
            \draw[->] (star_I_2.south) -- (S_star_I.north);
            \draw[->] (star_I_2.south) -- (I_star_I_1.north);

            \draw[->] (plus_R.south) -- (star_R_1.north);
            \draw[->] (star_R_1.south) -- (beta_star_R_1.north);
            \draw[->] (star_R_1.south) -- (I_star_R.north);
        \end{tikzpicture}
    \end{adjustbox}
\end{center}

que representa el sistema:

\begin{align*}
    S' & = - a_1 * S * I \\
    I' & = b_2 * S * I   \\
    R' & = a_4 * I
\end{align*}

\section{Determinación de las soluciones que pasan a la siguiente generación}

Dada una población en una generación, se toma un conjunto de individuos de esta y se mutan, y se toma otro subconjunto de individuos y se cruzan entre ellos. Todos estos individuos son seleccionados de forma aleatoria. Este nuevo subconjunto formado por la población inicial de la generación, los individuos resultantes de la mutación y los individuos resultantes de los cruzamientos tienen que ser filtrados para tomar una cantidad igual a la población inicial presente en la generación con el objetivo de poder repetir este proceso múltiples veces para poder realizar numerosas generaciones. Para poder filtrar este subconjunto se toman los individuos que mejor puntuaciones obtuvieron con respecto a los datos iniciales, o sea, se toman los sistemas que mejor ajustan a los datos