\begin{abstract}

    Los sistemas de ecuaciones diferenciales ordinarias lineales con respecto a los parámetros se pueden utilizar para predecir y describir fenómenos de la naturaleza, la ciencia, la tecnología y la sociedad. Esta predicción se realiza mediante la generación de modelos matemáticos a partir de observaciones del fenómeno. La obtención del modelo permite analizar la relación entre las distintas variables y parámetros que ocurren en el fenómeno observado.

    Para encontrar de forma automática modelos que desriben un conjunto de datos se desarrollan diversos métodos que son capaces de plantear de forma simbólica los sistemas que describen los datos. Estos algoritmos utilizan el método de regresión simbólica. La problema de regresión simbólica en sistemas de ecuaciones diferenciales se puede resolver mediante el uso de algoritmos genéticos o redes neuronales.

    En este trabajo se propone una herramienta para encontrar el sistema de ecuaciones diferenciales lineales con respecto a los parámetros que mejor describa un conjunto de datos utlizando regresión simbólica mediante algoritmos genéticos. Se aproximan los parámetros presentes en el modelo mediante la solución de un sistema de ecuaciones lineales y en caso de que los datos presenten ruido se utiliza un spline de suavizado para la aproximación de los datos. El comportamiento de la herrmienta propuesta muestra la capacidad de generación de un sistema que describe los datos, incluso ante la presencia de ruido.


\end{abstract}

\begin{enabstract}

    Ordinary differential equations systems linear with respect to parameters can be used to predict and describe phenomena in nature, science, technology, and society. This prediction is made by generating mathematical models from observations of the phenomenon. Obtaining the model allows to analyze the relationship between the different variables and parameters that occur in the observed phenomenon.

    To automatically find models that describe a set of data, various methods are developed that are capable of symbolically posing the systems that describe the data. These algorithms use the symbolic regression method. The symbolic regression problem in systems of differential equations can be solved by using genetic algorithms or neural networks.

    In this work, a tool is proposed to find the oridnary differential equations system linear with respect to the parameters that best describes a data set using symbolic regression through genetic algorithms. The parameters present in the model are approximated by solving a linear equations system and in case the data present noise, a smoothing spline is used to approximate the data. The behavior of the proposed tool shows the generation capacity of a system that describes the data, even in the presence of noise.

\end{enabstract}