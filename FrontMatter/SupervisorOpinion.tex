\begin{opinion}
    En este trabajo se propone un algoritmo génetico para encontrar el sistema de ecuaciones diferenciales lineal con respecto a los parámetros que mejor describa un conjunto de datos.

    Para realizar esta tesis Enrique tuvo que incursionar en temas que no están incluidos en su plan de estudios; resolver creativamente problemas que aparecieron como parte del proceso de desarrollo; analizar, valorar y desechar bibliotecas que contribuyeran a la solución del problema; y adquirir habilidades relacionadas con la escritura de documentos científicos y la presentación oral de los resultados.

    Durante este trabajo hemos sido testigos del talento, la capacidad de trabajo y la perserverancia de Kike, atributos estos que sumados a sus cualidades humanas (y su excelente tino en la selección de gatos para describir todo tipo de situaciones) hicieron de este tiempo de tesis una experiencia muy placentera, tanto desde el punto de vista profesional como personal.

    Por los resultados obtenidos, y por la forma en que se obtuvieron,  consideramos que estamos en presencia de un excelente trabajo realizado por un excelente profesional de la Ciencia de la Computación.

    % \vspace{1cm}


    \begin{flushright}
        \underline{\hspace{6.5cm}}\\
        MSc. Fernando Raul Rodriguez Flores

        Facultad de Matemática y Computación

        Universidad de la Habana

        Noviembre, 2022
    \end{flushright}

    \begin{flushright}
        \underline{\hspace{6.5cm}}\\
        Lic. Ernesto Alejandro Borrego Rodríguez

        Facultad de Matemática y Computación

        Universidad de la Habana

        Noviembre, 2022
    \end{flushright}

\end{opinion}

