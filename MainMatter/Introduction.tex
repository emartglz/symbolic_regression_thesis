\chapter*{Introducción}\label{chapter:introduction}
\addcontentsline{toc}{chapter}{Introducción}

\qquad

Tengo que ver que pongo aqui de regresión porque todo lo moví para los preliminares
% La regresión es un conjunto de procesos estadísticos para estimar las relaciones entre una o más variables dependientes y una o más variables independientes \cite{johnson2015applied}. La variable dependiente es la que es explicada mientras que la independiente es la que se utiliza para explicar la variación de la variable independiente. Se le llama regresión simple si solo se tienen dos variables, una dependiente y una independiente, en caso de que se posean más variables independiente se le llama regresión múltiple. Si como resultado de la regresión se obtiene un hiperplano entonces obtiene el nombre de regresión lineal \cite{mann2007introductory}. Por ejemplo, el método de los mínimos cuadrados ordinarios calcula el único hiperplano que minimiza la suma de las diferencias al cuadrado entre los datos dados y ese hiperplano, este es un tipo de regresión. Este método se puede expresar como encontrar los valores de $\beta$ que minizan $S$ dado el modelo $f$ y los puntos $(xi, yi)$ donde:

% $$S = \sum_{i=1}^{n}(y_i - f(x_i, \beta))^2$$

% Teniendo este hiperplano, el investigador puede estimar la variable dependiente cuando las variables independientes toman un conjunto dado de valores.

% El análisis de regresión se puede utilizar para la predicción y el pronóstico de datos y en algunas situaciones, se puede utilizar el análisis de regresión para inferir relaciones causales entre las variables independientes y dependientes \cite{mann2007introductory}.

Teniendo en cuenta que la regresión permite encontrar los valores que mejor ajustan un modelo dado conjunto de datos, se puede intentar utilizar la regresión para determinar de forma automática el sistema de ecuaciones diferenciales ordinarias lineal en los parámetros que mejor describe un conjunto de puntos de la forma $(t_i, y_i)$.

Una función es lineal en los parámetros si es de la forma:

$$f(t,y) = \sum_{i=1}^{n} a_i * g_i(t, y)$$

donde los $a_i$ son parámetros y todas las funciones $g_i(t,y)$ son funciones que dependen de la variable $t$, de la variable $y$, pero no dependen de ningún parámetro.

Esta definición se puede extender a sistemas de ecuaciones si todas las funciones cumplen esta propiedad.

Un ejemplo de una función lineal con respecto a los parámetros sería:

\begin{equation}
    \label{eqn:ode_example}
    I` = a * I^2
\end{equation}

Si se plantea que $y(t_i)$ es la solución de la ecuación diferencial evaluada en el punto $t_i$, entonces un indicador de cuán bien el sistema describe los valores de $(ti, yi)$ pudiera ser el valor $L$, donde:

$$L = \sum_{i=1}^{n} (y(t_i) - y_i)^2$$

Cuando se usa un valor como $L$ en el que se considera la suma de cuadrados de las diferencias, se dice que estamos en presencia de un problema de mínimos cuadrados, porque lo que se quiere es minimizar esa suma de cuadrados.

Entonces, buscar el sistema de ecuaciones diferenciales que mejor describe los datos, se reduce a buscar el sistema de ecuaciones diferenciales que haga que el valor de $L$ sea lo más pequeño posible.

\subsection*{Objetivos}

El objetivo general de este trabajo es diseñar e implementar una herramienta que permita encontrar, con muy poco esfuerzo del usuario, el sistema de ecuaciones diferenciales lineales en los parámetros que mejor ajuste un conjunto de datos de la forma $(ti, yi)$ con $1 < i < n$. Para ello, se trazaron los siguientes objetivos específicos:


\subsubsection*{Objetivos específicos}

\begin{enumerate}
    \item Enunciar las diferencias que plantea la regresión simbólica sobre el resto de regresiones existentes.
    \item Plantear qué son las metaheurísticas, específicamente los algoritmos genéticos.
    \item Evaluar la calidad de la herramienta implementada ante distintos modelos de ecuaciones diferenciales.
    \item Analizar los resultados obtenidos a través de un conjunto de métricas(AÚN NO SE CUÁLES MÉTRICAS VOY A UTILIZAR) y técnicas de visualización.
\end{enumerate}

\subsection*{Organización de la tesis}

Este documento está organizado en 4 capítulos que recogen las
distintas etapas por las que transitó la investigación.

En el capítulo 1 \textbf{Elementos de la regresión simbólica} se realiza una
introducción a los elementos y conceptos de esta área abordados a lo largo
del trabajo.
