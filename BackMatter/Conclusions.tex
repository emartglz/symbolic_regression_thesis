\chapter*{Conclusiones}\label{chapter:conclusions}
\addcontentsline{toc}{chapter}{Conclusiones}

En este trabajo se planteó un sistema de regresión simbólica basada en algoritmos genéticos para encontrar un sistema de ecuaciones diferenciales lineales con respecto a los parámetros en el que es posible determinar qué variables intervienen en cada ecuación.

Con este fin se definió cómo representar los sistemas de ecuaciones diferenciales en forma de árbol computacional de manera que solo se permitiese representar sistemas lineales con respecto a los parámetros. Para el uso del algoritmo genético se definieron las operaciones de mutación, cruzamiento y selección sobre estos árboles. Además se utilizó un spline de suavizado para eliminar el ruido en los datos.

Se utilizó la propiedad de linealidad con respecto a los parámetros de los sistemas de ecuaciones para ajustar los parámetros en cada solución resolviendo un sistema de ecuaciones lineales.

Para probar el funcionamiento del sistema creado se utilizaron datos generados a partir de 5 sistemas de ecucaciones diferenciales conocidos, comprobándose así la calidad de las soluciones obtenidas por la regresión simbólica.