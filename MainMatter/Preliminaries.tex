\chapter{Preliminares}\label{chapter:preliminaries}

El objetivo de este trabajo es encontrar el sistema de ecuaciones diferenciales lineales con respecto a los parámetros que mejor ajuste un conjunto de datos y para ello es necesario definir los conceptos de ecuaciones diferenciales, ecuaciones diferenciales lineales con respecto a los parámetros, regresión, ajuste de mínimos cuadrados, regresión simbólica, regresión simbólica en sistemas de ecuaciones diferenciales, algoritmos genéticos y splines. En este capítulo se presentan cada uno de estos elementos.
% En el capítulo anterior se definió el objetivo del presente trabajo, para ello se plantearon los conceptos de sistema de ecuaciones diferenciales lineales en los parámetros y la relación que existe entre ajustar un conjunto de datos y el algoritmo de mínimos cuadrados.

En la sección \ref{section:differential_equation_lineal_in_params} se presentan las ecuaciones diferenciales lineales con respecto a los parámetros. La regresión, ajuste mínimo cuadrático de datos y regresión simbólica se presentan en las secciones \ref{section:regression}, \ref{section:min_square} y \ref{section:symbolic_regression}. En la sección \ref{section:symbolic_regression_in_does} se describe cómo se aplica la regresión simbólica para encontrar el lado derecho de ecuaciones direrenciales. En la sección \ref{section:genetic_algorithm} se presentan los elementos fundamentales de los algoritmos genéticos. Finalmente en la sección \ref{section:smoothing_splines} se presentan los splines y los splines de suavizado.
% En la sección 1.1 de este capítulo se plantea lo que es una ecuación diferencial, en la sección siguiente se define de forma más detallada qué es una ecuación diferencial lineal con respecto a los parámetros. Una técnica que se utilizó para el desarrollo del presente trabajo es la regresión, los detalles se pueden encontrar en la sección \ref{section:regression}. Las características ajuste cuadrático y regresión simbólica aparecen en las dos secciones siguientes, respectivamente. A continuación en el capítulo se puede ver cómo se utiliza la regresión simbólica directamente en un sistema de ecuaciones diferenciales. En la última sección del presente capítulo se detalla qué es un algoritmo genético y se plantean ejemplos de situaciones en las que es útil esta metaheurística.

% A continuación se plantea que es una ecuación diferencial y se muestra un ejemplo de este tipo de ecuaciones.

% \section{Ecuaciones diferenciales}\label{section:differential_equation}
\section{Ecuaciones diferenciales lineales con respecto a los parámetros}\label{section:differential_equation_lineal_in_params}

La comprensión de los fenómenos que ocurren en la naturaleza y la sociedad está relacionanda con la Matemática y la Física, en particular, con las ecuaciones diferenciales. La modelación a través de ecuaciones diferenciales posibilita tanto la descripción como la predicción del desarrollo de fenómenos. La transferencia de calor \cite{p-transferencia-calor}, la transferencia de masa \cite{p-transferencia-masa}, el desarrollo de una población \cite{p-desarrollo-poblacion}, las relaciones amorosas \cite{p-amor}, el desarrollo de enfermedades como el VIH-SIDA \cite{p-desarrollo-vih}, el cólera \cite{p-desarrollo-colera}, la peste bubónica \cite{p-desarrollo-peste} o el COVID-19 \cite{p-desarrollo-covid} se estudian utilizando sistemas de ecuaciones diferenciales.

Una ecuación diferencial se define como una ecuación que contiene las derivadas de una o más variables dependientes, con respecto a una o más variables independientes \cite{gaucel2014learning}. Una ecuación diferencial ordinaria (EDO) es una que contiene derivadas en función de una sola variable (por ejemplo, el tiempo). La forma general de una ecuación diferencial ordinaria es:

$$y'(t)=f(t, y(t)) \qquad y(t_0) = y_0$$

donde $y(t)$ es una función y $y_0$ es una condición inicial. A $f(t, y(t))$ se le llama parte derecha de la ecuación diferencial.

La solución de una ecuación diferencial es una función que al ser sustituida en la ecuación hace que se satisfaga dicha ecuación. Si además, la ecuación diferencial es ordinaria, su parte derecha es continua en un intervalo cerrado y se tienen condiciones iniciales, entonces la solución de dicha ecuación diferencial existe y es única \cite{zill2012first}. Un ejemplo de ecuación diferencial ordinaria es:

$$y' = x * y^{\frac{1}{2}}$$

en donde la solución sería

$$y = \frac{1}{16} * x^4.$$

Una ecuación diferencial puede tener parámetros, de ser así la función $f$ se escribe como $f(t, y(t), a)$ donde $a$ es un parámetro. Un ejemplo de una ecuación diferencial con parámetros es:

$$y' = a * x * y^{\frac{1}{2}}.$$

% \section{Ecuaciones diferenciales lineales con respecto a los parámetros}\label{section:differential_equation_lineal_in_params}

Una ecuación diferencial es lineal con respecto a los parámetro si es de la forma

$$\frac{dX_i}{dt} = \sum_{i=1}^{n} a_i * f_i(t, y(t))$$

donde los $a_i$ son parámetros y todas las funciones $f_i(t,y)$ dependen de la variable $t$, de la variable $y$, pero no dependen de ningún parámetro $a_i$.

Esta definición se puede extender a sistemas de ecuaciones diferenciales si todas las ecuaciones cumplen esta propiedad. Que el sistema sea lineal con repecto a los parámetros permite modelar múltiples sistemas dinámicos poblacionales como el sistema de Lotka Volterra \cite{Hoppensteadt:2006}:

$$X' = \alpha * X - \beta * X * Y$$
$$Y' = \delta * X * Y - \gamma * Y$$

En ocasiones solo se poseen muestras de algún fenómeno físico en concreto y no se conocen las ecuaciones que lo describen. Para determinar qué sistema de ecuaciones diferenciales refleja el fenómeno se utiliza un tipo de regresión.

\section{Regresión}\label{section:regression}

La regresión es un conjunto de procesos estadísticos para estimar las relaciones entre una o más variables dependientes y una o más variables independientes. La relación que se busca se modela en forma de función paramétrica escogida de antemano por el investigador \cite{johnson2015applied}.

Dado un conjunto de $n$ muestras de $m$ variables independientes $x_i$ y el correspondiente valor de las $o$ variables dependientes $y_i$, $\{(x_i, y_i)\}^n_{i=1} \subset \mathbb{R}^m \times \mathbb{R}^o$, un vector de $k$ parámetros $\Theta \subset \mathbb{R}^k$ y una función $F : \mathbb{R}^{k + m} \rightarrow \mathbb{R}^o$ se define una función de ajuste como una función $E(y_i, f_i)$, donde $y_i$ es el valor de las variables dependientes observado con las variables independientes $x_i$ y $f_i = F(\Theta, x_i)$ es el valor estimado.

El problema de la regresión consiste en \textit{ajustar los datos}: encontrar los valores de los $k$ parámetros $\theta_1, \theta_2, \dots, \theta_k$ que minimicen la función de ajuste $E$
\begin{equation*}
    e = min_{\theta \in \Theta} \; E(y_i, f_i),
\end{equation*}

donde $e$ es el error de ajuste \cite{statisticintroductions}. Mientras más pequeño es el valor de $e$, mejor se considera el ajuste. En la siguiente sección se muestra un ejemplo de regresión.


% Se explica la variable dependiente mientras que la independiente es la que se utiliza para explicar la variación de la variable dependiente. Se le llama regresión simple si solo se tienen dos variables, una dependiente y una independiente, en caso de que se posean más variables independientes se le llama regresión múltiple \cite{mann2007introductory}.

% El análisis de regresión se puede utilizar para la predicción y el pronóstico de datos y en algunas situaciones, se puede utilizar el análisis de regresión para inferir relaciones causales entre las variables independientes y dependientes \cite{mann2007introductory}.

% La primera forma de regresión fue el método de mínimos cuadrados, que fue publicado por Legendre en 1805, y por Gauss en 1809. Legendre y Gauss aplicaron el método al problema de determinar, a partir de observaciones astronómicas, las órbitas de los cuerpos alrededor del Sol, en su mayoría cometas, pero también más tarde los planetas menores recién descubiertos.

\section{Ajuste mínimo cuadrático de datos}\label{section:min_square}

El método de mínimos cuadrados es un tipo de regresión que se utiliza para aproximar la solución de modelos en los que se conoce la forma general de la función que describe los datos observados, pero no los parámetros de esta función conocida.

El algoritmo de ajuste mínimo cuadrático se puede expresar como: resolver el problema de optimización de encontrar los valores de $\theta$ que minimizan $S$ dado el modelo $f$ y los puntos $(xi, yi)$ donde:

$$S = \sum_{i=1}^{n}(y_i - f(x_i, \theta))^2.$$

Una vez se resuelve el problema de optimización, se obtiene el vector de parámetros $\theta$ y se puede estimar la variable dependiente cuando las variables independientes toman un conjunto dado de valores. El ajuste mínimo cuadrático de datos se utiliza cuando se conoce la forma general de la función que describe los datos observados, pero hay ocasiones en la práctica en las que la función no se conoce, por ejemplo, el sistema de ecuaciones diferenciales que describe el virus COVID-19. Para encontrar la función que describe los datos observados se puede utilizar la regresión simbólica.

\section{Regresión simbólica}\label{section:symbolic_regression}

La regresión simbólica consiste en encontrar una expresión matemática, en forma simbólica, que ajuste con el error que se desee $e$ un conjunto de muestras. La introducción de la regresión simbólica generalmente se atribuye a John R. Koza \cite{zelinka2005analytic}. Koza mostró que la regresión simbólica puede usarse para descubrir modelos mediante la codificacion de expresiones matemáticas como árboles computacionales. En tales árboles, los nodos internos representan funciones ($+$, $-$, $*$, etc) que se extraen de un conjunto predeterminado de posibilidades, y los nodos hojas representan variables o constantes ($x_1$, $x_2$, $\dots$, $-1$, $\pi$, etc). Los parámetros de los modelos usualmente se calculaban en el propio algoritmo o cada cierto tiempo se resolvía un problema de optimización volviendo computacionalmente costoso el método. Un ejemplo de un árbol computacional que planteó Koza es:

\begin{center}
    \begin{adjustbox}{width=0.35\textwidth, keepaspectratio}
        \begin{tikzpicture}[
                roundnode/.style={circle, draw, fill=gray!25, very thick, minimum size=7mm},
                squarednode/.style={rectangle, draw, fill=gray!25, very thick, minimum size=5mm},
            ]
            %Nodes
            \node[roundnode]      (plus)                             {$+$};
            \node[roundnode]           (star1)   [below left=of plus]    {$*$};
            \node[squarednode]         (a_1)   [below left=of star1]    {$a_1$};
            \node[squarednode]         (y_1)     [below=of star1]         {$y_1$};
            \node[roundnode]           (star2)   [below right=of plus]   {$*$};
            \node[squarednode]         (a_2)    [below left=of star2]    {$a_2$};
            \node[roundnode]           (neg)     [below=of star2]         {$-$};
            \node[roundnode]           (star3)   [below=of neg]         {$*$};
            \node[squarednode]         (y_1_2)   [below=of star3]   {$y_1$};
            \node[squarednode]         (y_2)     [below right=of star3]   {$y_2$};

            %Lines
            \draw[->] (plus.south) -- (star1.north);
            \draw[->] (plus.south) -- (star2.north);
            \draw[->] (star1.south) -- (a_1.north);
            \draw[->] (star1.south) -- (y_1.north);
            \draw[->] (star2.south) -- (a_2.north);
            \draw[->] (star2.south) -- (neg.north);
            \draw[->] (neg.south) -- (star3.north);
            \draw[->] (star3.south) -- (y_1_2.north);
            \draw[->] (star3.south) -- (y_2.north);
        \end{tikzpicture}%
    \end{adjustbox}
\end{center}

que representa la expresión:

$$a_1 * y_1 + a_2 * -(y_1 * y_2)$$

Al no requerir una especificación a priori de un modelo, la regresión simbólica no se afecta por el desconocimiento de la estructura del modelo. El método intenta descubrir las relaciones presentes en el conjunto de datos. Para ello, se prueban múltiples modelos posibles evaluando su calidad con respecto a alguna métrica de interés, en lugar de imponer una estructura de modelo que se considere matemáticamente manejable desde una perspectiva humana.

Para evaluar que tan cercano está el modelo obtenido con respecto al modelo original se define una función de ajuste. La función de ajuste hace que los resultados obtenidos a lo largo de la regresión simbólica se acerquen más a la solución deseada ya que tiene en cuenta no solo las métricas de error (valores que definen cuán cerca están los resultados obtenidos a los resultados deseados), sino cualquier métrica que desee definir el usuario con el objetivo de obtener un resultado con características específicas, por ejemplo modelos más pequeños o con menor cantidad de parámetros. Esto facilita el análisis posterior de los resultados al permitir, en el modelo obtenido, asociar algunos parámetros a significados de la vida real, por ejemplo la cantidad de individuos que mueren en cada instante de tiempo en el sistema de lotka volterra.

La regresión simbólica tiene la desventaja de además de ser un problema NP-difícil \cite{virgolin2022symbolic}, tener un espacio de búsqueda mucho más grande que otros tipos de problemas de ajustes de datos. Por ejemplo, tanto en la regresión lineal como no lineal, el espacio de búsqueda es $\mathbb{R}^m$ y en la regresión simbólica se pueden explorar todas las funciones que van de $\mathbb{R}^m$ a $\mathbb{R}^n$.

Sin embargo, la característica de poseer un espacio de búsqueda mayor a otros métodos también tiene ventajas, ya que el resultado pueden ser múltiples modelos y sus correspondientes conjuntos de parámetros. Examinar la colección de modelos resultantes permite al usuario identificar una solución que se ajuste mejor a algunas características en particular. Por ejemplo, que el modelo posea ecuaciones con poca cantidad de parámetros o que la diferencia entre los datos evaluados en el modelo y los datos originales sea menor que un error específico.

Existen dos métodos que se pueden utilizar en la regresión simbólica. El primero se llama evolución gramatical y el segundo algoritmos genéticos \cite{zelinka2005analytic}. Se desarrollan diversos sofwares que utilizan la regresión simbólica con el fin de obtener la función que mejor ajusta un conjunto de datos que describen un fenómeno. Dentro de los softwares utilizados para la regresión simbólica se encuentran \textit{gplearn} que es una biblioteca de código abierto desarrollada en el lenguaje de programación \textit{Python} \cite{gplearn}, \textit{Eureqa} es un sofware  comercial \cite{schmidt2013eureqa} y \textit{AI Feynman} es un método que utiliza un acercamiento mediante el uso de la técnica divide y vencerás \cite{udrescu2020ai}.

En el año 1994, John R. Koza planteó que la regresión simbólica se podía utilizar con múltiples objetivos como generar programas de forma automática, crear árboles de decisión de clasificación y crear funciones que generen números aleatorios de alta entropía. También Koza definió que se podía utilizar la regresión simbólica para encontrar sistemas de ecuaciones diferenciales de forma automática.\cite{koza1994genetic}

\section{Regresión simbólica para EDOs}\label{section:symbolic_regression_in_does}

Los sistemas de ecuaciones diferenciales permiten modelar fenómenos que ocurren en la sociedad y en la naturaleza. Permitiendo describir y predecir el desarrollo de estos fenómenos. En ocasiones no se conoce con precisión el modelo de ecuaciones que está relacionado con los datos que describen al fenómeno, y es por esto que se necesita un mecanismo para encontrar el modelo.

En el año 2008 se utilizó la regresión simbólica junto con el algoritmo de mínimos cuadrados para encontrar el sistema de ecuaciones diferenciales lineales en los parámetros que mejor ajustase un conjunto de datos \cite{iba2008inference}.

En el año 2014, un grupo de investigadores utilizaron la regresión simbólica para encontrar sistemas dinámicos. El acercamiento que utilizaron consistía en reducir el problema de regresión simbólica para sistemas de múltiples ecuaciones diferenciales en el problema de regresión simbólica para un sistema de una sola ecuación diferencial. De esta forma generaban para cada ecuación del sistema un grupo de ecuaciones y luego probaban distintas combinaciones de las ecuaciones de los subconjuntos obtenidos hasta encontrar la que mejor aproximase el conjunto de datos \cite{gaucel2014learning}.

La investigación que se realizó en el año 2008 se utilizó en otro estudio correspondiente al año 2019. Los objetivos de ambas investigaciones son el mismo, en el estudio más reciente se sustituye el método de mínimos cuadrados para encontrar los parámetros por un algoritmo de descenso por gradientes \cite{kronberger2019identification}.

% Por un largo tiempo, la regresión simbólica solo era de dominio de los seres humanos, pero los estudios muestran cómo en las últimas décadas también se ha convertido en el dominio de los ordenadores. En la actualidad existen dos métodos que se utilizan en la regresión simbólica por medio de ordenadores. El primero se llama evolución gramatical y el segundo algoritmos genéticos \cite{zelinka2005analytic}. El último método se explicará en la siguiente sección.

El acercamiento utilizado en la investigación del año 2019 utilizaba la regresión simbólica mdiante el uso de algoritmos genéticos. Esta técnica se emplea de igual manera en la propuesta de solución del presente trabajo. En la siguiente sección se describen los algoritmos genéticos.

\section{Algoritmos genéticos}\label{section:genetic_algorithm}

Un algoritmo genético es una metaheurística inspirada en el proceso de selección natural \cite{mitchell1998introduction}. En este tipo de algoritmos se generan soluciones de alta calidad en problemas de optimización y búsqueda. Es un método para pasar de una población de ``cromosomas'' o individuos (por ejemplo, cadenas de unos y ceros, palabras o árboles computacionales) a una nueva población mediante el uso de ``selección natural'' junto con los operadores inspirados en la genética de cruzamiento, mutación y selección. Cada cromosoma consta de ``genes'' o características (por ejemplo, bits, letras o nodos).

Los algoritmos genéticos se usan para resolver problemas de optimización de la forma $min f(x)$, donde $x$ pertenece a un conjunto $C$ dado. Se le llama solución a cualquier $x$ de $C$. En dependencia de la estructura de $C$, las soluciones pueden tener distintas formas, por ejemplo vectores de números reales como en los problemas de optimización de dimensión finita \cite{mitchell1998introduction} o pueden ser árboles como en el caso de la regresión simbólica \cite{mitchell1998introduction}.

El algoritmo genético es un proceso iterativo, en el que se toma una población inicial de individuos y se le aplican modificaciones generando nuevos individuos. De la población resultante se seleccionan algunos sujetos que pasarán a ser la población inicial de la siguiente iteración del método. La población de la primera iteración se genera creando individuos aleatoriamente. A cada iteración del algoritmo se le llama generación.

El proceso de creación de nuevas generaciones se repite hasta que se alcanza una condición de parada. Algunas condiciones de parada comunes son \cite{mitchell1998introduction}:

\begin{itemize}
    \item Se encuentra un individuo lo suficientemente cercano a la solución del problema.
    \item Se alcanza el número máximo fijado de generaciones.
    \item Se alcanza la cantidad de tiempo o cómputo máximo asignado
    \item La cercanía de los individuos respecto a la solución del problema está alcanzando o ha alcanzado un nivel tal que las iteraciones sucesivas ya no producen mejores resultados.
    \item Combinaciones de las anteriores.
\end{itemize}

Para el proceso de modificación de la población de una generación se utilizan dos operadores: la mutación y el cruzamiento. Las operaciones se realizan sobre individuos aleatorios seleccionados de la población. La mutación realiza cambios aleatorios en alguna característica de un individuo seleccionado, y el cruzamiento intercambia características aleatorias de dos individuos seleccionados, obteniendo dos nuevos individuos. Existen operaciones de cruzamiento en la que solo se obtiene un nuevo individuo como resultado de la operación, este tipo de cruzaminento se utiliza en la propuesta de solución planteada en este trabajo.

Una vez culminado el proceso de modificación de la población mediante el uso de las operaciones de mutación y cruzamiento, se pasa a la operación de selección. La operación escoge de la población inicial de la generación y del conjunto de individuos resultantes de las mutaciones y cruzamientos aquellos que se acerquen más a la solución del problema. También se escogen un conjunto de individuos aleatorios con el fin de no caer en mínimos locales en la búsqueda de la solución que más se acerque a la solución del problema \cite{mitchell1998introduction}. En total, el proceso de selección escoge una cantidad de individuos igual a la población inicial de la generación.

Las operaciones de cruzamiento, mutación y selección dan como resultado la población de soluciones de la próxima generación. Dado que los individuos más cercanos a la solución del problema fueron escogidos en la operación de selección, la distancia del mejor individuo de la población a la solución nunca aumenta entre cada generación.

% Con el fin de explorar de maneras distintas el espacio de búsqueda, vale la pena ajustar parámetros del algoritmo genético como la probabilidad de mutación, de cruzamiento y el tamaño de la población, para encontrar configuraciones adecuadas para la clase de problema en la que se trabaja. Los parámetros típicamente interactúan entre sí de forma no lineal, por lo que no se puede optimizar uno a la vez. Hay mucha discusión y enfoques para la selección de parámetros de los algoritmos genéticos en la literatura de computación evolutiva, pero no hay resultados concluyentes sobre lo que es mejor; la mayoría de los investigadores usan lo que funcionó bien en casos previamente informados \cite{mitchell1998introduction}.

Si un conjunto de datos $\{x_i, Y_i\}$ que describen el fenómeno $f$ es de la forma $Y_i = f(x_i) + \epsilon _i$ donde los $\epsilon _i$ son valores aleatorios, independientes y con media 0, entonces se dice que el conjunto de datos posee ruido. Los datos que se utilizan para encontrar una solución en el algoritmo genético pueden poseer ruido. Altos niveles de ruido pueden ocasionar que el modelo seleccionado como solución no ajuste de forma correcta otros datos que describan el mismo fenómeno pero que no posean ruido. Para eliminar el término $\epsilon _i$ en los datos (eliminar el ruido en los datos) se pueden utilizan varias técnicas, una de estas técnicas son los Splines de suavizado.

\section{Smoothing de suavizado}\label{section:smoothing_splines}

Un spline es una función definida por intervalos, donde la función de cada intervalo es un polinomio. El máximo grado de los polinomios que se utilizan define el grado del spline \cite{ahlberg1967theory}. Los extremos de los intervalos se llaman ``nudos'' y el spline pasa por cada uno de los nudos seleccionados.

Un spline de suavizado es un spline en el que los nudos son todos los puntos dados pero la curva resultante no pasa necesariamente por los nudos. El spline de suavizado cúbico $\hat{f}$ que aproxima el fenómeno $f$ se define como el resultado del problema de minimización:

$$\sum_{i=1}^n (Y_i - \hat{f}(x_i))^2 + \lambda \int \hat{f}''(x)^2 dx,$$

donde $\{x_i, Y_i\}$ es un conjunto de puntos con ruido que describe a $f$ y $\lambda$ se define como factor de suavizado y es mayor igual que 0. Si el factor de suavizado es 0, entonces el spline de suavizado coincide con el el spline donde se escogen como nudos todos los datos. A medida que el valor de $\lambda$ aumenta, el resultado se acerca a la aproximación de los datos por el método de mínimos cuadrados \cite{green1993nonparametric}.

En este capítulo se definieron conceptos utilizados en la propuesta de solución. Esta propuesta utiliza un spline de suavizado para la obtención de una aproximación de las derivadas de un conjunto de datos dado. Una vez se tienen las aproximaciones de las derivadas, se utiliza la regresión simbólica mediante un algoritmo genético para la obtención de un sistema de ecuaciones diferenciales lineales con respecto a los parámetros que ajuste el conjunto de datos. Para usar un algoritmo genético se definieron varios elementos:

\begin{itemize}
    \item Cuáles son las posibles soluciones
    \item Cómo aplicar un operador de cruzamiento
    \item Cómo aplicar un operador de mutación
    \item Cómo determinar cuán buena es una solución
    \item Cómo determinar qué soluciones pasan a las próximas generaciones
\end{itemize}

En el próximo capítulo se describe cómo se aplican estos elementos a la regresión simbólica.
